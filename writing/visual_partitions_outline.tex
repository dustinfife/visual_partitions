% Options for packages loaded elsewhere
\PassOptionsToPackage{unicode}{hyperref}
\PassOptionsToPackage{hyphens}{url}
%
\documentclass[
  english,
  doc,floatsintext]{apa6}
\usepackage{amsmath,amssymb}
\usepackage{lmodern}
\usepackage{ifxetex,ifluatex}
\ifnum 0\ifxetex 1\fi\ifluatex 1\fi=0 % if pdftex
  \usepackage[T1]{fontenc}
  \usepackage[utf8]{inputenc}
  \usepackage{textcomp} % provide euro and other symbols
\else % if luatex or xetex
  \usepackage{unicode-math}
  \defaultfontfeatures{Scale=MatchLowercase}
  \defaultfontfeatures[\rmfamily]{Ligatures=TeX,Scale=1}
\fi
% Use upquote if available, for straight quotes in verbatim environments
\IfFileExists{upquote.sty}{\usepackage{upquote}}{}
\IfFileExists{microtype.sty}{% use microtype if available
  \usepackage[]{microtype}
  \UseMicrotypeSet[protrusion]{basicmath} % disable protrusion for tt fonts
}{}
\makeatletter
\@ifundefined{KOMAClassName}{% if non-KOMA class
  \IfFileExists{parskip.sty}{%
    \usepackage{parskip}
  }{% else
    \setlength{\parindent}{0pt}
    \setlength{\parskip}{6pt plus 2pt minus 1pt}}
}{% if KOMA class
  \KOMAoptions{parskip=half}}
\makeatother
\usepackage{xcolor}
\IfFileExists{xurl.sty}{\usepackage{xurl}}{} % add URL line breaks if available
\IfFileExists{bookmark.sty}{\usepackage{bookmark}}{\usepackage{hyperref}}
\hypersetup{
  pdftitle={Visual Partitioning for Multivariate Models: An Approach for Identifying and Visualizing Complex Multivariate Datasets},
  pdfauthor={Dustin A. Fife1 \& Jorge L. Mendoza2},
  pdflang={en-EN},
  hidelinks,
  pdfcreator={LaTeX via pandoc}}
\urlstyle{same} % disable monospaced font for URLs
\usepackage{graphicx}
\makeatletter
\def\maxwidth{\ifdim\Gin@nat@width>\linewidth\linewidth\else\Gin@nat@width\fi}
\def\maxheight{\ifdim\Gin@nat@height>\textheight\textheight\else\Gin@nat@height\fi}
\makeatother
% Scale images if necessary, so that they will not overflow the page
% margins by default, and it is still possible to overwrite the defaults
% using explicit options in \includegraphics[width, height, ...]{}
\setkeys{Gin}{width=\maxwidth,height=\maxheight,keepaspectratio}
% Set default figure placement to htbp
\makeatletter
\def\fps@figure{htbp}
\makeatother
\setlength{\emergencystretch}{3em} % prevent overfull lines
\providecommand{\tightlist}{%
  \setlength{\itemsep}{0pt}\setlength{\parskip}{0pt}}
\setcounter{secnumdepth}{-\maxdimen} % remove section numbering
% Make \paragraph and \subparagraph free-standing
\ifx\paragraph\undefined\else
  \let\oldparagraph\paragraph
  \renewcommand{\paragraph}[1]{\oldparagraph{#1}\mbox{}}
\fi
\ifx\subparagraph\undefined\else
  \let\oldsubparagraph\subparagraph
  \renewcommand{\subparagraph}[1]{\oldsubparagraph{#1}\mbox{}}
\fi
% Manuscript styling
\usepackage{upgreek}
\captionsetup{font=singlespacing,justification=justified}

% Table formatting
\usepackage{longtable}
\usepackage{lscape}
% \usepackage[counterclockwise]{rotating}   % Landscape page setup for large tables
\usepackage{multirow}		% Table styling
\usepackage{tabularx}		% Control Column width
\usepackage[flushleft]{threeparttable}	% Allows for three part tables with a specified notes section
\usepackage{threeparttablex}            % Lets threeparttable work with longtable

% Create new environments so endfloat can handle them
% \newenvironment{ltable}
%   {\begin{landscape}\begin{center}\begin{threeparttable}}
%   {\end{threeparttable}\end{center}\end{landscape}}
\newenvironment{lltable}{\begin{landscape}\begin{center}\begin{ThreePartTable}}{\end{ThreePartTable}\end{center}\end{landscape}}

% Enables adjusting longtable caption width to table width
% Solution found at http://golatex.de/longtable-mit-caption-so-breit-wie-die-tabelle-t15767.html
\makeatletter
\newcommand\LastLTentrywidth{1em}
\newlength\longtablewidth
\setlength{\longtablewidth}{1in}
\newcommand{\getlongtablewidth}{\begingroup \ifcsname LT@\roman{LT@tables}\endcsname \global\longtablewidth=0pt \renewcommand{\LT@entry}[2]{\global\advance\longtablewidth by ##2\relax\gdef\LastLTentrywidth{##2}}\@nameuse{LT@\roman{LT@tables}} \fi \endgroup}

% \setlength{\parindent}{0.5in}
% \setlength{\parskip}{0pt plus 0pt minus 0pt}

% \usepackage{etoolbox}
\makeatletter
\patchcmd{\HyOrg@maketitle}
  {\section{\normalfont\normalsize\abstractname}}
  {\section*{\normalfont\normalsize\abstractname}}
  {}{\typeout{Failed to patch abstract.}}
\patchcmd{\HyOrg@maketitle}
  {\section{\protect\normalfont{\@title}}}
  {\section*{\protect\normalfont{\@title}}}
  {}{\typeout{Failed to patch title.}}
\makeatother
\shorttitle{Visual Partitioning}
\usepackage{csquotes}
\usepackage{amsmath}
\usepackage{tikz}
\usetikzlibrary{shapes.geometric,arrows, positioning}
\usepackage[LGRgreek]{mathastext}
\ifxetex
  % Load polyglossia as late as possible: uses bidi with RTL langages (e.g. Hebrew, Arabic)
  \usepackage{polyglossia}
  \setmainlanguage[]{english}
\else
  \usepackage[main=english]{babel}
% get rid of language-specific shorthands (see #6817):
\let\LanguageShortHands\languageshorthands
\def\languageshorthands#1{}
\fi
\ifluatex
  \usepackage{selnolig}  % disable illegal ligatures
\fi

\title{Visual Partitioning for Multivariate Models: An Approach for Identifying and Visualizing Complex Multivariate Datasets}
\author{Dustin A. Fife\textsuperscript{1} \& Jorge L. Mendoza\textsuperscript{2}}
\date{}


\affiliation{\vspace{0.5cm}\textsuperscript{1} Rowan University\\\textsuperscript{2} University of Oklahoma}

\abstract{
Users of statistics quite frequently use multivariate models to make conditional inferences (e.g., stress affects depression, after controlling for gender). These inferences are often done without adequately considering (or understanding) the assumptions one makes when claiming these inferences. Of particular concern is when there are unmodeled nonlinear and/or interaction effects. With such unmodeled multiplicative effects, inferences based on a main effects model are not merited. On the other hand, when these effects are properly modeled, complex multivariate analyses can be ``partitioned'' into distinct components to ease interpretation. In this paper, we highlight when conditional inferences are contaminated by other features of the model and identify the conditions under which effects can be partitioned. We also reveal a strategy for partitioning multivariate effects into uncontaminated blocks using visualizations. This approach simplifies multivariate analyses immensely, without oversimplifying the analysis.
}



\begin{document}
\maketitle

\hypertarget{introduction}{%
\section{Introduction}\label{introduction}}

\begin{itemize}
\tightlist
\item
  Suppose we were to peruse journal and find the results shown in Table \ref{tab:anovatab}

  \begin{itemize}
  \tightlist
  \item
    How are we to make sense of this?
  \item
    Does health increase ideation or decrease it?
  \item
    What is the nature of the interaction?
  \end{itemize}
\end{itemize}

\begin{table}[tbp]

\begin{center}
\begin{threeparttable}

\caption{\label{tab:anovatab}ANOVA Summary Table of the Suicide Ideation Analysis}

\begin{tabular}{lccccc}
\toprule
 & DF & SS & MS & F & p\\
\midrule
stress & 1 & 1,598.37 & 1,598.37 & 874.68 & <0.001\\
$\text{stress}^2$ & 1 & 8,374.25 & 8,374.25 & 4582.65 & <0.001\\
health & 1 & 2,097.66 & 2,097.66 & 1147.9 & <0.001\\
friend ideation & 1 & 699.67 & 699.67 & 382.88 & <0.001\\
depression & 1 & 5,797.63 & 5,797.63 & 3172.64 & <0.001\\
friend ideation $ \times $ depression & 1 & 2,954.06 & 2,954.06 & 1616.55 & <0.001\\
Residuals & 2993 & 5,469.36 & 1.83 &  & \\
\bottomrule
\end{tabular}

\end{threeparttable}
\end{center}

\end{table}

\begin{itemize}
\tightlist
\item
  ANOVA summary tables are painfully uninformative
\item
  Even if we knew direction/nature, there's still the multivariate nature to contend with

  \begin{itemize}
  \tightlist
  \item
    What can be interpreted in isolation?
  \item
    What must be interpreted multivariately?
  \end{itemize}
\item
  This paper introduces ``visual partitions''

  \begin{itemize}
  \tightlist
  \item
    Supplement to ANOVA summary tables
  \item
    Visuals that sucinctly communicate the nature of a multivariate analysis
  \item
    These partitions can be interpreted in relative isolation without worrying about misinterpreting multivariate model
  \end{itemize}
\end{itemize}

\hypertarget{tools}{%
\section{Tools}\label{tools}}

\begin{itemize}
\tightlist
\item
  Flexplot
\item
  Partial Residual Plots/AVPs
\item
  Marginal Plots
\end{itemize}

\hypertarget{assumptions-and-visual-partitions}{%
\section{Assumptions and Visual Partitions}\label{assumptions-and-visual-partitions}}

\begin{itemize}
\tightlist
\item
  Assume for each analysis there is a ``true'' model

  \begin{itemize}
  \tightlist
  \item
    Researcher may or may not have discovered the true model
  \item
    May instead have a ``hypothesized model''
  \end{itemize}
\item
  Suppose ``true model'' only contains main effects

  \begin{itemize}
  \tightlist
  \item
    Simple bivariate plots may be misleading (e.g., suppressor effects)
  \item
    Rather, an AVP or a PRP would be appropriate
  \item
    But, then marginal effects can be visualized in isolation
  \end{itemize}
\item
  Suppose ``true model'' contains both main effects and nonlinear effects

  \begin{itemize}
  \tightlist
  \item
    Linear plots will be misleading
  \item
    AVPs (or PRPs) must include the nonlinear component
  \end{itemize}
\end{itemize}

\hypertarget{visual-partitions-in-confirmatory-research}{%
\section{Visual Partitions in Confirmatory Research}\label{visual-partitions-in-confirmatory-research}}

\hypertarget{visual-partitions-in-exploratory-research}{%
\section{Visual Partitions in Exploratory Research}\label{visual-partitions-in-exploratory-research}}

\hypertarget{xx-step-strategy}{%
\subsection{XX Step Strategy}\label{xx-step-strategy}}

\hypertarget{example-analysis}{%
\section{Example Analysis}\label{example-analysis}}

\pagebreak

\hypertarget{references}{%
\section{References}\label{references}}


\end{document}
